\documentclass[a4paper, 12pt]{article}

\usepackage[spanish]{babel}
\usepackage[utf8]{inputenc}
\usepackage{amsmath}
\usepackage{a4wide}

\title{Examen Métodos Numéricos I \\ 2ª Semana Febrero 2013}
\author{Unai Aguilera Irazabal DNI: 45663055-M}

\begin{document}
\maketitle

\section*{Problema 1}
\subsection*{Eliminación gaussiana}
A partir del sistema de ecuaciones se construye la matriz aumentada de
coeficientes del sistema.

\begin{align*}
\begin{pmatrix}
  4 & 1 & 1 & 4\\
  1 & 4 & -2 & 4 \\
  3 & 2 & -4 & 6
\end{pmatrix}
\end{align*}

A partir de ella se aplica el proceso de eliminaciones para convertir la 
matriz en una matriz triangular superior.

\begin{align*}
\begin{pmatrix}
  4 & 1 & 1 & 4\\
  1 & 4 & -2 & 4 \\
  3 & 2 & -4 & 6
\end{pmatrix} 
\xrightarrow[R_3 = -\frac{3}{4}R_1 + R_3]{R_2 = -\frac{1}{4}R_1 + R_2}
\begin{pmatrix}
  4 & 1 & 1 & 4\\
  0 & 15/4 & -9/4 & 3 \\
  0 & 5/4 & -19/4 & 3
\end{pmatrix}
\end{align*}

\begin{align*}
\xrightarrow{R_3 = -\frac{1}{3}R_2 + R_3}
\begin{pmatrix}
  4 & 1 & 1 & 4\\
  0 & 15/4 & -9/4 & 3 \\
  0 & 0 & -4 & 2
\end{pmatrix}
\end{align*}

Ahora se aplica una substitución hacia atrás para obtener los valores de las
incógnitas

\begin{align*}
-4I_3 = 2
\rightarrow
I_3 = -\frac{1}{2}
\end{align*}

\begin{align*}
\frac{15}{4}I_2 - \frac{9}{4}I_3 = 3 
\rightarrow
\frac{15}{4}I_2 - \frac{9}{4}(-\frac{1}{2}) = 3
\rightarrow
I_2 = \frac{1}{2}
\end{align*}

\begin{align*}
4I_1 + I_2 + I_3 = 4 
\rightarrow
4I_1 + (\frac{1}{2}) + (-\frac{1}{2}) = 4
\rightarrow
I_1 = 1
\end{align*}

\subsection*{Método de reducción de Crout}

\section*{Problema 2}

\section*{Problema 3}

\section*{Problema 4}

\end{document}