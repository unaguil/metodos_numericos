\documentclass[a4paper, 12pt]{article}

\usepackage[spanish]{babel}
\usepackage[utf8]{inputenc}
\usepackage{amsmath}
\usepackage{a4wide}

\title{Examen Métodos Numéricos I \\ 2ª Semana Febrero 2013}
\author{Unai Aguilera Irazabal DNI: 45663055-M}

\begin{document}
\maketitle

\section*{Problema 1}
\subsection*{Eliminación gaussiana}
A partir del sistema de ecuaciones se construye la matriz aumentada de
coeficientes del sistema.

\begin{align*}
\begin{pmatrix}
  4 & 1 & 1 & 4\\
  1 & 4 & -2 & 4 \\
  3 & 2 & -4 & 6
\end{pmatrix}
\end{align*}

A partir de ella se aplica el proceso de eliminaciones para convertir la 
matriz en una matriz triangular superior.

\begin{align*}
\begin{pmatrix}
  4 & 1 & 1 & 4\\
  1 & 4 & -2 & 4 \\
  3 & 2 & -4 & 6
\end{pmatrix} 
\xrightarrow[R_3 = -\frac{3}{4}R_1 + R_3]{R_2 = -\frac{1}{4}R_1 + R_2}
\begin{pmatrix}
  4 & 1 & 1 & 4\\
  0 & 15/4 & -9/4 & 3 \\
  0 & 5/4 & -19/4 & 3
\end{pmatrix}
\end{align*}

\begin{align*}
\xrightarrow{R_3 = -\frac{1}{3}R_2 + R_3}
\begin{pmatrix}
  4 & 1 & 1 & 4\\
  0 & 15/4 & -9/4 & 3 \\
  0 & 0 & -4 & 2
\end{pmatrix}
\end{align*}

Ahora se aplica una substitución hacia atrás para obtener los valores de las
incógnitas

\begin{align*}
-4I_3 = 2
\rightarrow
I_3 = -\frac{1}{2}
\end{align*}

\begin{align*}
\frac{15}{4}I_2 - \frac{9}{4}I_3 = 3 
\rightarrow
\frac{15}{4}I_2 - \frac{9}{4}(-\frac{1}{2}) = 3
\rightarrow
I_2 = \frac{1}{2}
\end{align*}

\begin{align*}
4I_1 + I_2 + I_3 = 4 
\rightarrow
4I_1 + (\frac{1}{2}) + (-\frac{1}{2}) = 4
\rightarrow
I_1 = 1
\end{align*}

\subsection*{Método de reducción de Crout}
Se plantea la descomposición LU de la matriz de coeficientes A de tal forma que

\begin{align*}
\begin{pmatrix}
  l_{11} & 0 & 0\\
  l_{21} & l_{22} & 0\\
  l_{31} & l_{32} & l_{33}
\end{pmatrix} 
\begin{pmatrix}
  1 & u_{12} & u_{13}\\
  0 & 1 & u_{23}\\
  0 & 0 & 1
\end{pmatrix}
=
\begin{pmatrix}
  a_{11} & a_{12} & a_{13}\\
  a_{21} & a_{22} & a_{23}\\
  a_{31} & a_{32} & a_{33}
\end{pmatrix}
\end{align*}

Al multiplicar las filas de L por la primera columna de U se obtiene que
\begin{align*}
l_{11} = 4;~~~
l_{21} = 1;~~~
l_{31} = 3;~~~
\end{align*}

ahora, al multiplicar la primera fila de L por las columnas de U
\begin{align*}
u_{12} = \frac{a_{12}}{l_{11}} = \frac{1}{4};~~~
u_{13} = \frac{a_{13}}{l_{11}} = \frac{1}{4};~~~
\end{align*}

posteriormente se multiplican las filas de L por la segunda columna de U
\begin{align*}
l_{22} = a_{22} - l_{21}u_{12} = 4 - 1\frac{1}{4} = \frac{15}{4}\\
l_{32} = a_{32} - l_{31}u_{12} = 2 - 3\frac{1}{4} = \frac{5}{4}
\end{align*}

la segunda fila de L por las columnas de U
\begin{align*}
u_{23} = \frac{a_{23}}{l_{23}u_{13}} =
\frac{-2 - 1\frac{1}{4}}{\frac{15}{4}} = -\frac{3}{5}
\end{align*}

y finalmente, las filas de L por la tercera columna de U
\begin{align*}
l_{33} = a_{33} - l_{31}u_{13} - l_{32}u_{23} =
-4 - 3\frac{1}{4} - \frac{5}{4}(-\frac{3}{5}) = -4
\end{align*}

Se obtiene así la siguiente descomposición LU de la matriz A
\begin{align*}
\begin{pmatrix}
  4 & 0 & 0\\
  1 & 15/4 & 0\\
  3 & 5/4 & -4
\end{pmatrix} 
\begin{pmatrix}
  1 & 1/4 & 1/4\\
  0 & 1 & -3/5\\
  0 & 0 & 1
\end{pmatrix}
=
\begin{pmatrix}
  4 & 1 & 1\\
  1 & 4 & -2\\
  3 & 2 & -4
\end{pmatrix}
\end{align*}

Ahora se amplía la matriz L con la columna de términos independientes del
sistema

\begin{align*}
\begin{pmatrix}
  4 & 0 & 0 & 4\\
  1 & 15/4 & 0 & 4 \\
  3 & 5/4 & -4 & 6
\end{pmatrix}
\end{align*}

y se realiza una substitución hacia delante para transformar los terminos
independientes

\begin{align*}
4b_1 = 4
\rightarrow
b_1 = 1
\end{align*}

\begin{align*}
b_1 + \frac{15}{4}b_2 = 4 
\rightarrow
1 + \frac{15}{4}b_2 = 4
\rightarrow
b_2 = \frac{12}{15}
\end{align*}

\begin{align*}
3b_1 + \frac{5}{4}b_2 + 4b_3 = 6 
\rightarrow
3 + \frac{5}{4}(\frac{12}{15}) + 4b_3 = 6
\rightarrow
b_3 = -\frac{1}{2}
\end{align*}

Utilizando dichos términos transformados para ampliar la matriz U

\begin{align*}
\begin{pmatrix}
  1 & 1/4 & 1/4 & 1\\
  0 & 1 & -3/5 & 12/15 \\
  0 & 0 & 1 & -1/2
\end{pmatrix}
\end{align*}

y de donde realizando una substicución hacia atrás se obtienen los valores de
las incógnitas
\begin{align*}
I_3 = -\frac{1}{2}
\end{align*}

\begin{align*}
I_2 - \frac{3}{5}I_3 = \frac{12}{15} 
\rightarrow
I_2 - \frac{3}{5}(-\frac{1}{2}) = \frac{12}{15}
\rightarrow
I_2 = \frac{1}{2}
\end{align*}

\begin{align*}
I_1 + \frac{1}{4}I_2 + \frac{1}{4}I_3 = 1 
\rightarrow
I_1 + \frac{1}{4}(\frac{1}{2}) + \frac{1}{4}(-\frac{1}{2}) = 1 
\rightarrow
I_1 = 1
\end{align*}

\section*{Problema 2}
Para aplicar la cuadratura gaussiana a 

\begin{equation*}
	\int^{1}_{0} x e^{-x^2} dx
\end{equation*}

es necesario cambiar el intervalo de integración a (-1, 1). Para ello se aplica
el siguiente cambio de variable

\begin{equation*}
x = \frac{(b - a)t + b + a}{2} = \frac{(1 - 0)t + 1 + 0}{2} = \frac{t + 1}{2}
\end{equation*}

\begin{equation*}
dx = \frac{b - a}{2} = \frac{(1 - 0)}{2} = \frac{1}{2}
\end{equation*}

\begin{equation*}
	\int^{1}_{0} x e^{-x^2} dx =
	\frac{1}{4}\int^{1}_{-1} (t + 1) e^{-\frac{(t + 1)^2}{4}} dx
\end{equation*}

El valor de la integral obtenido de manera analítica es

\begin{equation*}
  \int^1_0 x e^{-x^2} = -\frac{1}{2} \int^{-1}_0 e^z dz =
  -\frac{1}{2} e^z|^{-1}_{0} = \frac{1}{2}(1-\frac{1}{e}) = 0.316060
\end{equation*}

\section*{Problema 3}

\section*{Problema 4}

\end{document}